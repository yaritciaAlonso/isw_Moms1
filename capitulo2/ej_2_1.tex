
\subsection*{2.1 Discuta el inpacto de las interfaces de usuario sobre confiable}

\textbf{Yaritcia Alonso González:}
 Esta peculiaridad del software es a menudo mal utilizada. Aunque es posible cambiar ciertas cosas de las que
ya están hechas para satisfacer las nuevas necesidades que surgan conforme al uso que vaya surgiendo
 este cambio nunca se realiza a la ligera, y ciertamente no se intenta sin antes
reelaborar el proyecto y verificar el impacto del cambio en profundidad. al
a la inversa, a menudo se requiere que los ingenieros de software realicen cambios sustanciales.
La ductilidad del software lleva a pensar que hacer cambios es trivial, pero en la práctica
no es asi, ya que al hacer los cambios  se tienen que pensar bien porque un error puede ser fatal.

\textbf{Lucero Nava Santos:}
En las disciplinas tradicionales de ingeniería, el ingeniero tiene herramientas para describir
Las cualidades del producto de una manera distinta del producto en sí. en la ingeniería
de software esta distinción no es tan clara ya que las cualidades de un producto de software
a menudo se mezclan en especificaciones, junto con las cualidades que tenga el producto, es por eso
que tiene que ser más específico a la hora de ralizar algun cambio.
\begin{quote}

\end{quote}


